%%%%%%%%%%%%%%%%%%%%%%%%%%%%%%%%%%%%%%%%%
% "ModernCV" CV and Cover Letter
% LaTeX Template
% Version 1.3 (29/10/16)
%
% This template has been downloaded from:
% http://www.LaTeXTemplates.com
%
% Original author:
% Xavier Danaux (xdanaux@gmail.com) with modifications by:
% Vel (vel@latextemplates.com)
%
% License:
% CC BY-NC-SA 3.0 (http://creativecommons.org/licenses/by-nc-sa/3.0/)
%
% Important note:
% This template requires the moderncv.cls and .sty files to be in the same 
% directory as this .tex file. These files provide the resume style and themes 
% used for structuring the document.
%
%%%%%%%%%%%%%%%%%%%%%%%%%%%%%%%%%%%%%%%%%

%----------------------------------------------------------------------------------------
%	PACKAGES AND OTHER DOCUMENT CONFIGURATIONS
%----------------------------------------------------------------------------------------

\documentclass[11pt,a4paper,sans]{moderncv} % Font sizes: 10, 11, or 12; paper sizes: a4paper, letterpaper, a5paper, legalpaper, executivepaper or landscape; font families: sans or roman

\moderncvstyle{classic} % CV theme - options include: 'casual' (default), 'classic', 'oldstyle' and 'banking'
\moderncvcolor{green} % CV color - options include: 'blue' (default), 'orange', 'green', 'red', 'purple', 'grey' and 'black'

\usepackage{lipsum} % Used for inserting dummy 'Lorem ipsum' text into the template


\usepackage[scale=0.75]{geometry} % Reduce document margins
\setlength{\hintscolumnwidth}{3.4cm} % Uncomment to change the width of the dates column
%\setlength{\makecvtitlenamewidth}{10cm} % For the 'classic' style, uncomment to adjust the width of the space allocated to your name

%----------------------------------------------------------------------------------------
%	NAME AND CONTACT INFORMATION SECTION
%----------------------------------------------------------------------------------------

\firstname{Valentin} % Your first name
\familyname{Kopylov} % Your last name

% All information in this block is optional, comment out any lines you don't need
\title{Curriculum Vitae}
%\address{Molodezhnaya st., 3}{Moscow, Russia, 119296}
\phone{+7 (999) 825 72 66}
%\mobile{(000) 111 1112}
%\fax{(000) 111 1113}
\email{wally.kopylov@gmail.com}
%\homepage{staff.org.edu/~jsmith}{staff.org.edu/$\sim$jsmith} % The first argument is the url for the clickable link, the second argument is the url displayed in the template - this allows special characters to be displayed such as the tilde in this example
%\extrainfo{additional information}
\photo[80pt][0.2pt]{my} % The first bracket is the picture height, the second is the thickness of the frame around the picture (0pt for no frame)
%\quote{"A witty and playful quotation" - John Smith}

%----------------------------------------------------------------------------------------

\begin{document}

%----------------------------------------------------------------------------------------
%	COVER LETTER
%----------------------------------------------------------------------------------------

% To remove the cover letter, comment out this entire block

\clearpage

\recipient{The Laboratory for Comparative Social Research}{} % Letter recipient
\date{\today} % Letter date
\opening{Greetings,} % Opening greeting
\closing{Sincerely yours,} % Closing phrase
\enclosure[Attached]{curriculum vit\ae{}} % List of enclosed documents

%\makelettertitle % Print letter title

%My name is Valentin Kopylov, and I am a 3rd-year Bachelor's student, majoring in Political Science (NRU HSE). I am applying for the internship in your laboratory in order to obtain more practical research experience, expand my knowledge and to commence a  journey as a researcher in the sphere of social sciences.

%Throughout my first 2 years of study, I applied to various winter and summer schools, connected to the field of social sciences and quantitative methods applied within it, searching for my sphere of interest and research. Last year within a course paper I studied the factors, which contribute to the modification of European identity, based on multilevel regression analysis, and additionally, in order to construct the European identity index, I used the Principal Component Analysis and its features. Therefore, I tried to understand and conceptualise European identity and deduce the factor, which plays its role in the modification process. 

%In this year's course paper I switched the field of study to political trust and its determinants. My question of interest now is the role of the political regime (possibly as a moderator) in the relationship between the subjective evaluation of economic performance and political trust. Therefore, I'm planning to use a multilevel mixed-effects model in order to model the moderation of political regime across diverse countries and to consider 2 levels of analysis: both individual- and country-level. Worth mentioning that one of the main issues of the political trust research recently was the question of the measurement invariance of the political trust across countries, which is assumed to exist as an assumption. In my work, I'm trying to check for it using the Factor Analysis (actually, now I'm here). I would say, that my whole education on the program was deeply connected with the quantitative methodological path: as a student of the "political analysis" route, throughout my education, I was particularly interested in the applications of modern methods of quantitative social research. 

%The most important reason, why I would like to apply for the internship in your laboratory is that I am curious about trying myself in a real-world work connected to the research in the social sciences. And I am seeking to gain new experience and knowledge first of all and to establish my vector of professional pathway.

%I want to thank you in advance for the provided opportunities by your laboratory!



%\makeletterclosing % Print letter signature

\newpage

%----------------------------------------------------------------------------------------
%	CURRICULUM VITAE
%----------------------------------------------------------------------------------------

\makecvtitle % Print the CV title

%----------------------------------------------------------------------------------------
%	EDUCATION SECTION
%----------------------------------------------------------------------------------------


\section{Personal information}

\cvitem{}{Date of birth: 28.07.1999}
\cvitem{}{Place of birth: Moscow}

\section{Education}

\cventry{2017 -- present time}{Bachelor of Political Science}{ \newline National Research University -- Higher School of Economics}{Moscow}{}{}

\section{Work experience}

\cventry{May -- October 2020}{Research Assistant}{\newline The Laboratory for Comparative Social Research \newline (LCSR NRU HSE)}{Moscow}{}{}

%\cventry{2011--2012}{Masters of Commerce}{The University of California}{Berkeley}{\textit{GPA -- 8.0}}{First Class Honours}  % Arguments not required can be left empty
%\cventry{2007--2010}{Bachelor of Business Studies}{The University of California}{Berkeley}{\textit{GPA -- 7.5}}{Specialized in Commerce}

\section{Additional education}

\cvitem{Summer and winter schools:}{Summer University Prague 2019 \newline \textbf{\href{https://drive.google.com/open?id=1JnxlgsZ6LvxNMLB31psK7sX6zYgmrPQ0}{European identity between Unity and Diversity}} \newline Charles University \newline (Prague, Czech Republic, 7.09 -- 21.09.2019)}

\cvitem{}{IPSA -- HSE 2019 \newline International Summer School for Methods of Political \& Social Research:  \newline \textbf{\href{https://drive.google.com/open?id=1yNjB8j2GKXyZqtUkf-vChJnCE4hAK5yd}{Bayesian Statistics}} \newline (St. Petersburg, Russia, 8.08 -- 17.08.2019)}


\cvitem{}{\href{https://ecpr.eu/Events/PanelDetails.aspx?PanelID=8445&EventID=131}{ECPR 2019}  \newline \textbf{\href{https://drive.google.com/open?id=1RkLT8XN8F2OPQAijhZ8ANkdorTCjmBvH}{Advanced Topics in Applied Regression}} \newline Central European University \newline (Budapest, Hungary, 4.08 -- 9.08.2019)}

\cvitem{}{\href{https://escapes.hse.ru/mnd_place}{Escapes from modernity:}  \newline \textbf{\href{https://drive.google.com/open?id=1sVoovZCjU_Hw_KyXQBo9vXCvGajicS71}{Democratic Constitutional Design:} Negotiating Civil Engagement, Institutional Control and the Common Good} \newline NRU HSE \& University of Iceland \newline (Reykjavik, Iceland, 20.07 -- 25.07.2019)}

\cvitem{}{\href{https://lcsr.hse.ru/summer_school/summer2018/}{7th LCSR Summer School:} \newline \textbf{\href{https://drive.google.com/open?id=1ge84hHovxy8o7wIc2XNGnA3O8eLvvhSH}{Bayesian Approach in Social Science}} \newline (Moscow, Russia, 20.08 -- 31.08.2018)}

\cvitem{}{IPSA -- HSE 2018 \newline International Summer School for Methods of Political \& Social Research:  \newline \textbf{\href{https://drive.google.com/open?id=1d7211UYEkuRkHQj4JUGLrU5fCKuPafLv}{Regression Analysis}} \newline (St. Petersburg, Russia, 25.07 -- 10.08.2018)}


\cvitem{}{\href{https://escapes.hse.ru/mnv_place}{Escapes from modernity:}  \newline \textbf{\href{https://drive.google.com/open?id=12y60FKDSPzg91VhU_Y7FO23urZMNELzx}{Politics of Violence in post-Soviet states and societies}} \newline (Ilia State University, Georgia, 31.03 -- 6.04.2018)}


\cvitem{}{\href{https://escapes.hse.ru/pm_place}{Escapes from modernity:} \newline \textbf{\href{https://drive.google.com/open?id=1bY4jB8SunfeMpZYxg8sU7LceZPa4zfQe}{People and territories: Biopower and Geopolitics in the Black sea and the Caucasus}} \newline (University of Tartu, Estonia, 11.02 -- 17.02.2018)}





\section{Assistance}

\cvitem{Project Assistant}{\href{https://escapes.hse.ru/mnd_place}{Escapes from modernity:}  \newline \textbf{Democratic Constitutional Design: Negotiating Civil Engagement, Institutional Control and the Common Good} \newline NRU HSE \& University of Iceland}

\cvitem{Teaching Assistant}{\textbf{\href{https://www.hse.ru/en/ba/political/courses/219865348.html}{Mathematics and Statistics}} \newline  NRU HSE: <<Political Science>> program, 1-year \newline 2019}

\cvitem{Teaching Assistant}{\textbf{\href{https://www.hse.ru/en/ba/soc/courses/325422333.html}{Data Analysis in R}} \newline  NRU HSE: <<Sociology>> program, 4-year \newline 2019}

\cvitem{Teaching Assistant}{\textbf{\href{https://www.hse.ru/en/ba/economics/courses/292669321.html}{Data Science}} \newline  NRU HSE: <<Economics>> program, 2-year \newline 2020}

\cvitem{Teaching Assistant}{\textbf{\href{https://www.hse.ru/en/ba/political/courses/292686662.html}{Probability Theory and Mathematical Statistics}} \newline  NRU HSE: <<Political Science>> program, 2-year \newline 2020}

\cvitem{Teaching Assistant}{\textbf{\href{https://www.hse.ru/en/ba/political/courses/292675674.html}{An Introduction to Probability Theory and Mathematical Statistics II}} \newline  NRU HSE: <<Political Science>> program, 2-year \newline 2020}

\newpage 
\section{Volunteering}

\cvitem{}{\href{http://www.ism.uw.edu.pl/en/narva-xviii-international-student-research-conference-neglecting-the-borders-6-dimensions-of-eu-rus-relations-2/}{XVIII International Student Research Conference.} \newline <<\textbf{\href{https://drive.google.com/open?id=1m38C19Uk-WT5pkl-KFyefdNQEEN_JE08}{Neglecting the borders: 6Dimensions of EU-RUS relations}}>> \newline (University of Tartu, Estonia, 19.04 -- 21.04.2018)}

%----------------------------------------------------------------------------------------
%	AWARDS SECTION
%----------------------------------------------------------------------------------------
\section{Awards and achievements}

\cvitem{\href{https://drive.google.com/file/d/17i_T5Q3tPHaQlcGJTnqTS6d6T7LioWzW/view?usp=sharing}{Best technical solution}}{\href{https://www.hse.ru/dataculture/hackathon}{Data Culture Hack:} \newline Open Data Hackaton in social sciences,\newline HSE 2019}


\section{Course Papers}

\cvitem{Title}{\emph{(2020) Political Regime as a Moderator Between Political Trust and Subjective Evaluation of Economic Performance }}
\cvitem{Abstract}{The paper addresses the role of political regime as a moderator in a relationship between the subjective evaluation of economic performance and political trust based on Afrobarometer survey. In introduction, author outlines the role of political trust as an indicator of political legitimacy of state and proceeds to its major determinant - economic performance. However, the impact of economic performance varies across different political regimes, which leads to the potential discrepancy of this liaison. In theoretical part, author investigates different conceptualisations of political trust and captures the mechanisms of economic performance impact, concluding by remarks concerning role of political regime in their relationship. Further, author elaborates on cross-national measurement equivalence issue and proceeds to factor-analytic model of political trust, which achieves metric equivalence, allowing to compare the effects across different populations. Finally, author verifies the significant moderating role of political regime by applying regression analysis and further robustness checks.}
\cvitem{Mark}{\emph{8}}

\cvitem{}{}


\cvitem{Title}{\emph{(2019) European identity: Factors of Change}}
\cvitem{Abstract}{The article aims at the empirical re-operationalisation of the European identity concept through the construction of the empirically-valid concept based on the most relevant repeated cross-national survey research, such as European Values Study (and it's new 2017 issue), European Social Survey and Eurobarometer. The author tries to compile different shades of European identity through the questions of these particular surveys using PCA. Following the process of the construction of this concept, the author verifies several hypotheses about the role of the perception of the European institutions on the modification of the European identity, which occurred within the period 2010-2016, using multilevel mixed-effect model.}
\cvitem{Mark}{\emph{9}}

%\section{Masters Thesis}

%\cvitem{Title}{\emph{Money Is The Root Of All Evil -- Or Is It?}}
%\cvitem{Supervisors}{Professor James Smith \& Associate Professor Jane Smith}
%\cvitem{Description}{This thesis explored the idea that money has been the cause of untold anguish and suffering in the world. I found that it has, in fact, not.}

%----------------------------------------------------------------------------------------
%	WORK EXPERIENCE SECTION
%----------------------------------------------------------------------------------------

%\section{Experience}




%----------------------------------------------------------------------------------------
%	COMPUTER SKILLS SECTION
%----------------------------------------------------------------------------------------

\section{Computer skills}

\cvitem{Data analysis}{\textbf{R} \newline (Tidyverse: dplyr, ggplot2, tidyr, etc), \newline \newline \textbf{Python} \newline (NumPy, Pandas, Scikit-learn, Matplotlib, PyTorch), \newline \newline \textbf{SQL}}
\cvitem{}{\href{https://github.com/wallykop}{GitHub}, \href{https://www.kaggle.com/wallykop/competitions}{Kaggle}}
\cvitem{{Publishing software}}{ \textbf{\LaTeX}}
\cvitem{Office software}{Microsoft Office}

%----------------------------------------------------------------------------------------
%	COMMUNICATION SKILLS SECTION
%----------------------------------------------------------------------------------------

%\section{Communication Skills}

%\cvitem{2010}{Oral Presentation at the California Business Conference}
%\cvitem{2009}{Poster at the Annual Business Conference in Oregon}

%----------------------------------------------------------------------------------------
%	LANGUAGES SECTION
%----------------------------------------------------------------------------------------

\section{Languages}

\cvitemwithcomment{Russian}{Mothertongue}{}
\cvitemwithcomment{English}{\href{https://drive.google.com/open?id=1M82EsiNx9vNerH4AwxD6UFfDWrWe2ZGC}{IELTS:} 7.5}{}
\cvitemwithcomment{French}{Basic}{}
\cvitemwithcomment{Latin}{Basic}{}


%----------------------------------------------------------------------------------------
%	INTERESTS SECTION
%----------------------------------------------------------------------------------------

%\section{Interests}

%\renewcommand{\listitemsymbol}{-~} % Changes the symbol used for lists

%\cvlistdoubleitem{Piano}{Chess}
%\cvlistdoubleitem{Cooking}{Dancing}
%\cvlistitem{Running}

%----------------------------------------------------------------------------------------

\vspace{13ex}

\textit{UPDATED: 15.10.2020}

\end{document}